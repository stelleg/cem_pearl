\begin{figure*}
\includegraphics[width=\linewidth/3]{figures/12.pdf}
\includegraphics[width=\linewidth/3]{figures/13.pdf}
\includegraphics[width=\linewidth/3]{figures/14.pdf}
\includegraphics[width=\linewidth/3]{figures/15.pdf}
\includegraphics[width=\linewidth/3]{figures/16.pdf}
\includegraphics[width=\linewidth/3]{figures/17.pdf}
\includegraphics[width=\linewidth/3]{figures/18.pdf}
\includegraphics[width=\linewidth/3]{figures/19.pdf}
\includegraphics[width=\linewidth/3]{figures/20.pdf}
\caption{An example sequence of machine states during the evaluation of the term
$(\lambda a.(\lambda b.b \; a) (\lambda c.c
\; a)) ((\lambda i.i) (\lambda j.j))$. Order is left to right, top to bottom.
The free heap location $f$ is left out to save space. Dotted lines denote
the pointer for the closure's environment at a cell, and solid lines denote the
environment continuation. For example, in the first state, the environment
defined at location 3 corresponds to $c = a \cdot a = (\lambda i.i) \lambda j.j \cdot
\bullet$ in the $K$ machine's environment definition. We see the update rule in
effect when }
\label{fig:states}
\end{figure*}
